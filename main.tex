\documentclass{llncs}
%\documentclass[10pt, conference, compsocconf]{IEEEtran}

% From ESORICS'13 CFP
% Submitted papers should be at most 16 pages (using 10-point font), excluding 
% the bibliography and well-marked appendices, and at most 20 pages total.

\usepackage{listings}
\usepackage{url}

\usepackage{amssymb}
\usepackage[usenames]{color}
\definecolor{lightred}{rgb}{1,0.8,0.8}
\newcommand{\todo}[1]{\colorbox{red}{\textcolor{white}{\sffamily\bfseries\scriptsize TODO}} \textcolor{red}{#1} \textcolor{red}{$\blacktriangleleft$}}


\title{Architectures for Inlining Security Monitors in Web Applications}

\author{Jonas Magazinius \and Daniel Hedin \and Andrei Sabelfeld}
\institute{Chalmers University of Technology, Gothenburg, Sweden}
%%%%%%%%%%%%%%%% Schedule %%%%%%%%%%%%%%%%%%%%%%%%
\if 0
March 28 - paper shipped
March 27 - final polish
March 25 - instantiation
March 22 - intro and conclusions
March 22 - the bulk of instantiation
March 22 - implementation
March 21 - architecture
March 20 - related work
March 12 - implementation half-way
March 11 - architecture first pass
\fi
%%%%%%%%%%%%%%%%%%%%%%%%%%%%%%%%%%%%%%%%%%%%%%%

% comment out for the final version
\pagestyle{plain}

\begin{document}


\maketitle

% comment out for the final version
\thispagestyle{plain}

\begin{abstract}
Securing JavaScript in the browser is an open and challenging
problem. Code from pervasive third-party JavaScript libraries exacerbates the
problem because it is executed with the same privileges as the code
that uses the libraries.
%
An additional complication is that the different stakeholders have
different interests in the security policies to be enforced
in web applications.
%
This paper focuses on securing JavaScript code by \emph{inlining}
security checks in the code before it is executed.
%
We achieve great flexibility in the deployment options by considering
security monitors implemented as security-enhanced JavaScript interpreters.
%
%A key ingredient to open up for such flexibility is security monitoring in the form of a
%security monitor for JavaScript, written itself in JavaScript.
We propose architectures for inlining security monitors for JavaScript: via
browser, via web proxy, and via suffix proxy (web service). 
%
These architectures offer great flexibility: they are parametric in the monitor itself,
and they provide freedom in the choice of where the monitor is
injected, allowing to serve the interests of the different stake
holders: the users, code developers, code
integrators, as well as the system and network administrators.
%
We report on experiments that demonstrate successful deployment of a JavaScript
information-flow monitor with the different architectures.
\end{abstract}


% \begin{IEEEkeywords}
% Web Security; Polyglot; Injection; Cross-domain.
% \end{IEEEkeywords}








% 1-2 pages
\section{Introduction}
\label{sec:intro}
JavaScript is at the heart of what defines the modern
browsing experience on the web. JavaScript enables dynamic and interactive
web pages as they are experienced by the users. Glued together,
JavaScript code from different sources provides
a rich execution platform. Reliance on third-party code is
pervasive~\cite{Nikiforakis+:CCS12}, with the range of included code from
format validation snippets, to helper
libraries as jQuery, to helper services as Google Analytics, and all the way to fully-fledged services as
Google Maps and Yahoo!\@ Maps.

\paragraph{Securing JavaScript}
Securing JavaScript in the browser is an open and challenging
problem. Third-party code inclusion exacerbates the
problem. The \emph{same-origin policy (SOP)}, enforced by the modern
browsers, allows free communication to the Internet origin of a given web page 
while it
places some restrictions on communication to Internet domains
outside the origin. However, once third-party code is
included in a web page, it is executed with the same privileges as the code
that uses the libraries. This gives rise to a number of attack possibilities,
such as location
hijacking, behavioral tracking, leaking cookies, and sniffing browsing history~\cite{Jang+:CCS10}. 

\paragraph{Security policy stakeholders}
An additional complication is that the different stakeholders have
different interests in the security policies to be enforced
in web applications. 
%
\emph{Users} might demand stronger guarantees than those
offered by SOP when it is not desired that sensitive information leaves
the browser. This makes sense in such popular web applications as
password strength checkers or loan
calculators.
%
\emph{Code developers} clearly have the interest in protecting the
secrets associated with the web application. For example, they might
allow access to the first-party cookie to code from third-party
services, like Google (this is needed for the proper functioning of
such services
as Google Analytics), but under the condition that no sensitive part
of the cookie is leaked to the third party.
%
\emph{Code integrators} might have different levels of trust to the
different integrated components, perhaps depending on the origin. It
is desirable to invoke different protection mechanisms for the
different code that is integrated into the web application.
%
Finally, \emph{system and network administrators} also have a
stake in the security goals. It might be desirable to configure the
system and/or network so that certain users are protected to a larger
extent or communication to certain web sites is restricted to a larger
extent than to the others.  

\paragraph{Secure inlining for JavaScript}
This paper proposes a novel approach to securing JavaScript in web
applications in the presence of multiple stakeholders.
%
We focus on securing JavaScript code by \emph{inlining}
security checks in the code before it is executed.
%
A key feature
of our approach is focusing on security monitors implemented as
security-enhanced JavaScript interpreters, written in JavaScript
themselves. This, seemingly bold, approach leverages two-fold
flexibility. First, having complete information about a given
execution, security-enhanced JavaScript interpreters are able to
enforce fine-grained security policies such as \emph{information-flow
security}~\cite{Sabelfeld:Myers:JSAC}. Second, because the monitor/interpreter is written itself in
JavaScript, we achieve great flexibility in the deployment options.

\paragraph{Architectures for inlinig security monitors}
As our main contribution,
we propose architectures for inlining security monitors for JavaScript: via
browser, via web proxy, and via suffix proxy (web service). 
%
These architectures offer great flexibility: they are parametric in the monitor itself,
and they provide freedom in the choice of where the monitor is
injected, allowing to serve the interests of the different stake
holders: the users, code developers, code
integrators, as well as the system and network administrators.

We note that our approach is general: it applies to arbitrary security
monitors, implemented as JavaScript interpreters. The 
 Narcissus~\cite{Narcissus} project provides a baseline JavaScript
 interpreter written in JavaScript, an excellent starting
 point for supporting versatile security policies.

When introducing reference monitoring, Anderson~\cite{Anderson:72}
identifies the following requirements:
\begin{itemize}
\item \emph{monitor integrity}: the monitor must be tamperproof,
\item \emph{complete mediation}: the monitor must be always invoked, and
\item \emph{soundness}: the monitor must enforce the desired
  security policy.
\end{itemize}
In addition, requirements of \emph{conservativeness} and
\emph{transparency} are often in place for reference monitors to
ensure that no new behaviors are added or removed by monitors when the
original program is secure in the first place.


\paragraph{Instantiation}
In order to illustrate the usefulness of the approach, we present an
instantiation of the architectures to enforce secure information flow
in JavaScript. Information-flow control for JavaScript allows tracking
fine-grained security policies for web applications. Typically,
information sources  and sinks are given sensitivity labels, for
example, corresponding to the different Internet origins. The goal of
information-flow enforcement is to prevent \emph{explicit flows}, via
direct information propagation by assignment commands, as well as
\emph{implicit flows} via the control flow in the program.

Our focus on information flow is justified by
the nature of the JavaScript attacks from the empirical
studies~\cite{Jang+:CCS10,Nikiforakis+:CCS12} that demonstrate the current security
practices fail to  prevent 
location
hijacking, behavioral tracking, leaking cookies, and sniffing browsing
history. Jang et al. ~\cite{Jang+:CCS10} report on both explicit and
implicit flows exploited in the empirical studies.
%
In addition, inlining by security-enhanced interpreting is a
particularly suitable choice for tracking information flow in
JavaScript, because alternative approaches to inlining suffer from
scalability problems, as discussed in Section~\ref{sec:related}. 

Our instantiation results demonstrate how to deploy
\emph{JSFlow}~\cite{Hedin:Sabelfeld:CSF12,JSFlow}, secure information-flow
monitor for JavaScript by Hedin et al., via
browser, via web proxy, and via suffix proxy (web service).
%
We report on security and performance experiments that demonstrate successful deployment of a JavaScript
information-flow monitor with the different architectures.


% A growing number of web pages provide not only content, but services to their
% users. In order to provide the service the provider requires the user to
% provide potentially sensitive information such as user credentials, and payment
% information. For such pages it is in the interest of the user that the
% sensitive information is only disclosed to the intended recipient.  Today, no
% browsers offer such guarantees. However, modern browsers allow the
% functionality of the browser to be enriched via \emph{extensions}. 


%\paragraph{Third-party code integration}
%\label{sec:mash}

%Describe mashups, difficulties in mashup security and information flows in mashups.

%4 pages 
\section{Architectures}
\label{sec:arch}

This section presents the different architectures for inlining
security monitors. We motivate each architecture by scenarios of
intended usage, describe the details of the architectures, report on
security considerations as well as the pros and cons for each choice.



%%%%%%%%%%%%%%%%%%%%%%%%%%%%%%%%%%%%%%%%%%%%%%%%%%%%%%%%%%%%%%%%%%%%%%%%%%%%%%%%

\subsection{Browser extension}

Many websites require the user to provide sensitive information, e.g., user
credentials, or payment information. It lies in the interest of the user to
ensure the security of the sensitive information.  Modern browsers allow for
the functionality of the browser to be enriched via \emph{extensions}. By
deploying the security monitor via a browser extension it is possible to enforce
properties not normally offered by browsers.  Regardless of whether the user is
private or corporate, browser extensions provide a simple 
install-once deployment method.


\paragraph{Description}

A browser extension is a program that is installed into the browser in order to
change or enrich the functionality of the browser.  The basic idea behind
deploying via an extension is to replace the JavaScript engine with the monitor.
(Recall that we consider only monitors that function as interpreters.)

Even though the internal JavaScript engine cannot actually be replaced the same
effect can be achieved by turning off the standard JavaScript engine, and have
the extension traverse the page when loaded and execute the JavaScript scripts
using the monitor. 
In addition to this, the extension must register the monitor as the target of
all events to ensure that the event handlers are run by the monitor. Failing to
do so will not compromise security; since the internal engine is turned off
the event handler would simply not run, preventing the loaded page from functioning
properly.

The most common extension implementation language is JavaScript. Given that
the monitor is written in JavaScript it is a simple matter to embed the 
monitor in an extension.
This method was pioneered by Zaphod~\cite{Zaphod}, a
FireFox extension, that replaces the standard JavaScript engine with the
experimental Narcissus~\cite{Narcissus} engine.


\paragraph{Security considerations}

The integrity of the method is guaranteed by virtue of the fact that the
JavaScript engine is turned off and all scripts are interpreted by the monitor.
Hence, the scripts are not active; rather, they are passed as data to the
monitor, and are only able to influence the execution environment implemented
by the monitor and not the general execution environment of the monitor itself.
Since all scripts are executed using the monitor complete mediation is implied.
In addition, this also implies that the deployment method is sound given that
the monitor is sound. 

As described above, browser extension are installed into the browser. In order
to be able to enrich the functionality of the browser, the extensions run with
the same privileges as the browser. This entails that the monitor will be
running with elevated privileges. Compared to the other methods of deployment
this means that a faulty monitor not only jeopardizes the property enforced by
the monitor, but might jeopardize the integrity of the entire browser.

\paragraph{Pros and cons}

\todo{Again, is this really true - one could just as well imagine the other 
  methods to provide the same}
Since the extension is installed on locally in the browser of the user, it has
the benefit of giving the user direct control over what security policies to 
enforce on the browsed pages.

The \emph{Document Object Model} (DOM) provides the tree model of HTML
documents used in browsers. The DOM is a complex, richly linked tree structure,
which offers many challenges.  One example of such a feature, that can be
handled with relative ease compared to other methods of deployment, is
\lstinline{innerHTML}. \lstinline{innerHTML} is a property on all DOM objects,
that allows the text representation of the object to be changed. Any such
change causes the content of the object to be changed by parsing the changed
property. For other deployment methods this means that the value written to
\lstinline{innerHTML} must be validated, and possibly changed before the write
is allowed to occur --- even though scripts added using \lstinline{innerHTML}
are not executed, event handlers might still be run.
%
For the extension based deployment this is not a problem.  Since the standard
JavaScript execution engine is turned off it is safe to write to the
\lstinline{innerHTML} property. The write will trigger the parser to rebuild
the content of the node, but no scripts will be executed, and the result can be
handled in a manner analogous to the scripts on a loading page.

Additionally, the extension is entirely independent of the method of transport,
since the extension can tell the browser to download the resource. Since the 
browser performs the operation the result
is indistinguishable from the standard operation of the browser, and, unlike other 
methods of deployment, it is unimportant whether the scripts
are referenced using HTTP or HTTPS.

Whether the requirements of conservativeness and transparency can be met
depends on when the scripts are executed and that all functionality provided by
the browser can be implemented by the monitor.

However, some features remain challenging for the extension deployment method.
First, consider the order in which scripts are executed.  When web pages are
loaded, the scripts of the pages are executed as they are encountered while
parsing the web page. This means that the DOM tree of the page might not have
been fully constructed when the scripts execute.  For scripts that interact
with the DOM tree this is important, in particular if the script injects nodes
into the tree. Differences in the state of the DOM tree can be detected by 
scripts at execution time. Hence, to guarantee conservativeness and transparency 
the execution of scripts must occur at the same times in the DOM tree construction
as they would have in the unmodified browser. With some effort this can be achieved
using DOM mutation events.

Another challenge pertaining to conservativeness and transparency is that some
functionality provided by the browser might be difficult or expensive to
provide.  One example of this is \lstinline{document.write}. The semantics of
\lstinline{document.write} is loosely specified~\cite{DOM:LVL2} saying that
\lstinline{document.write} writes a string into the current position of the
document.  Intuitively, \lstinline{document.write} writes into the character
stream that is fed to the HTML parser, which, for obvious reason can have
drastic effects on the parsing of the page. For security reasons extensions are
prohibited from calling \lstinline{document.write}. Hence, when run via an
extension the monitor cannot easily provide the full functionality of
\lstinline{document.write}.  Consider the following example, where
\lstinline{document.write} is used inside a div element.

\begin{verbatim}
  <div><script>document.write('</div><div>');</script></div>
\end{verbatim}

The result is two div elements, rather than the original one. Since the write
takes place during the parsing of the page, the end tag written by
\lstinline{document.write} is matched with the initial div tag, and the written
start tag is matched with the final end tag. In order to implement
\lstinline{document.write} the monitor would have to take the interaction with the
between the content written by \lstinline{document.write}, the already parsed parts
of the page and the remaining page into account. In essence, the monitor would
have to reimplement the parsing process.  However, the practical use of
\lstinline{document.write} is limited; the typical use case is to inject scripts
into the page while loading. Such cases can be relatively easily identified
from the string passed as a parameter, and the intended behavior can be
implemented by the monitor.



%%%%%%%%%%%%%%%%%%%%%%%%%%%%%%%%%%%%%%%%%%%%%%%%%%%%%%%%%%%%%%%%%%%%%%%%%%%%%%%%

\subsection{Web proxy}

\todo{Daniel: I'm not sure about the story how the secret info is in the browser}
In a corporate setting, it is essential for a company to protect 
sensitive information pertaining to the company, as well as their employees. 
The common use of intranet portals to facilitate easy sharing of information 
within the company puts this sensitive information in the web browser. 
Meanwhile, employees are using the same web browsers for accessing untrusted content on 
the Internet. Naturally, it is in the interest of the company to ensure that 
the sensitive information is not accessible outside the domain of the company.
In this distributed setting, a \emph{web proxy} is a convenient way of delivering 
the security monitor to all clients with little effort. 

A web proxy is also appealing to a private user, as it includes the monitor in all browsed pages without modifying 
the browser itself. By configuring the browser to use a web proxy, the user can ensure the security 
of sensitive data.

\paragraph{Description}
A proxy specific to relaying HTTP requests is often referred to as a web proxy.
All modern browsers support using web proxies to relay requests.
It allows all outgoing HTTP requests to be 
relayed through the proxy server. The web proxy will act as a man-in-the-middle, making 
requests on behalf of the client. In the process, the proxy has the 
opportunity of modifying both the request and the response. 
As a monitor deployment method the opportunity to rewrite the response is 
of interest. A web proxy is a convenient way to include the monitor in browsed pages without 
modifying the browser itself. The browser is configured to relay 
all requests through the proxy, and the proxy rewrites the responses to include 
the monitor in each page. Modifying the response makes the method more 
intrusive to the HTML content, but less intrusive to the user's browser. 

\todo{Strong wording; any. Later is in order to be effective is used}
For the monitor to have any effect, all scripts bundled with 
the page must be executed within the monitor. The scripts can either be inline, 
e.g., included as part of the HTML page, or external, e.g., referenced in the 
HTML page to be downloaded from an external source.
Apart from including the monitor in all browsed pages, all inline and external 
scripts must be identified and rewritten by the web proxy to direct execution to the monitor.
\todo{A call to the monitor hard to understand - better use 'use the monitor to execute'?}
This is achieved by wrapping all JavaScript source code, whether inline or external, in a call to the monitor.
In the web proxy, different rewriting rules apply depending on whether the 
requested content is JavaScript, e.g., external, or HTML, e.g., inline. 
\todo{Wrapped hard to understand}
External 
JavaScript content is simply wrapped, whereas 
for HTML the web proxy has to identify and wrap all occurrences of inline 
JavaScript, both script-tags and event handlers.


\paragraph{Security considerations}

As opposed to a browser extension, which replaces the 
JavaScript engine, the monitor will be executed by the engine of the browser. 
By including the monitor in the page, it is executed in the context of the page. Naturally 
this is the same execution context in which all scripts bundled with the 
page are executed. Whereas a 
browser extension intercepts execution of all scripts by replacing the JavaScript 
engine of the browser, a web proxy must identify all inline and 
external scripts and rewrite them to ensure that they are executed within the 
monitored execution context. Failing to do so, script code will execute 
along-side the monitor, giving it access to the sensitive information the 
monitor was intended to protect. 
\todo{Move this sentence up before last?}
This effort continues as the monitor executes. Various JavaScript features, such as 
\emph{document.write}, allow an arbitrary string to be 
interpreted and rendered as HTML. Any scripts present in the string will be 
executed upon interpretation. The monitor must account for this and again rewrite 
the string to direct execution to the monitor.

Special consideration is required for secure HTTP connections (HTTPS). In an HTTPS 
connection, a certificate containing the public key of the target domain is sent 
along with the response. The response is encrypted using the private key of the target 
domain and decrypted with the public key contained in the certificate. The 
certificate is issued by a certificate authority (CA)
that validates the certificate by signing the public key. Given that the user 
trusts the CA, it can also trust that the public key belongs 
to the target domain, and that the response has not been modified in transit.
This poses a challenge for a web proxy designed exactly to modify the response. The solution 
is for the proxy to act as a CA by generating a
new certificate for the target domain on-the-fly, and sign the 
public key of the certificate with the private key of the proxy. Naturally, 
the proxy must be trusted by the user, either because the key is added to 
the users list of trusted CAs, or signed by one that the user trusts.

Complete mediation is achieved since all requests are relayed through the proxy 
and all occurrences are wrapped to direct execution to the monitor.

\todo{Jonas: Need to expand.}

\paragraph{Pros and cons}
\todo{Mass config is used; even chalmers uses images, i.e., nothing is configured per machine anyhow - browser independence is really
the key?}
In a corporate environment there are several benefits of using a web proxy 
compared to a browser extension. As an example, a web proxy can be centrally 
administered and configured, whereas a browser extension requires individual installation 
and configuration on each computer. Hence, using this approach can significantly 
reduce administration efforts in a larger corporation. 
Another benefit of using a web proxy is that it is browser-independent. The 
rewriting technique is the same regardless of the browser used, hence, there is no 
need to adapt the web proxy to suit a particular browser, allowing 
users their choice of browser.
Neither is it intrusive to the browser, as the user is not required to modify the 
browser in order to take advantage of security benefits. 
%However, there 
%are certain drawbacks as well. 

In order for the monitor to be effective, the web proxy 
has to identify all scripts. The process of doing so may require additional 
resources and introduce additional 
latency in comparison to that of a usual web proxy.

As an individual user, the main benefits are browser independence and  
increased security without having to modify the browser. How the proxy is 
deployed determines the users influence over 
the policies enforced.
\todo{explain what locally means?}
If deployed locally, the user has full influence over 
how policies are to be used, while for a remotely deployed proxy, the user is 
in the hand of the proxy administrator. 

Typically a web proxy relays and includes the monitor in all browsed pages, 
trusted or not, will run monitored. With a browser extension a user can
configure certain trusted web pages to run unmonitored for performance reasons.


\subsection{Suffix proxy (service)}
%\todo{Jonas: Just found out that the "service" we developed is actually a suffix proxy, which actually could have been implemented in a simpler way using .htaccess request rewriting. Also found a way to redirect all ports to a single port using iptables. Perfect solution.}
%\begin{itemize}
%\item What is a web proxy
%\item How can a web proxy help in distributing the monitor
%\item 
%\item Delivering the associated cookies?
%\item Same-origin policy in respect to different design choices
%\item More intrusive in HTML content, rewrite links and external references 
%\item Browser independent? Depends on the monitor
%\item Not intrusive to the browser
%\item Even less cumbersome to setup, browse to server URL
%\item Server-side and client-side configuration possible?
%\item Less general, only browsed pages
%\end{itemize}
\todo{Daniel: general comment - this section is mostly generic; speaks very little about the monitor}
\todo{Users is one perspective, but  service providers is another
  important one}
Certain users want to distinguish between trusted and untrusted web sites.
A \emph{suffix proxy} particularly suits a user who wants to run certain untrusted 
pages monitored to prevent leakage.

\paragraph{Description}
A \emph{suffix proxy} is a specialized web proxy, with a different approach to relaying the request. 
Instead of the common practice of configuring the browser to relay all requests through the web proxy,
the suffix proxy makes use of DNS wildcards to redirect the request to it.
Typically, the user navigates to a 
web application associated with the proxy and enters the target URL, e.g., \emph{http://google.com/search?q=proxy}, in 
an input field. To direct the request to the proxy, the target domain name is altered
by appending the domain name of the proxy, making the target domain a sub-domain of the proxy domain, e.g., \emph{http://google.com.proxy.domain/search?q=proxy}. 
The browser is then navigated to the modified URL.
The suffix proxy is set up such
that all requests to any sub-domain of the proxy domain are directed to the proxy domain, 
e.g., in DNS terms \emph{*.proxy.domain $\Rightarrow$ proxy.domain}. 
A web application on the proxy domain is set up to listen for such sub-domain requests.
When a request for a sub-domain is registered, it is intercepted by the web application.
The web application strips the proxy domain from the URL, leaving the original target URL, 
and makes the request on behalf of the client. As in the case of the web proxy, 
relaying the request to the target URL gives the suffix proxy an opportunity to modify and 
include the monitor in the response.



%The latter implies that apart from rewriting scripts within a page, all URLs to 
%external resources, i.e., to origins outside the scope of the original origin, 
%must be rewritten to include the domain of the proxy. This ensures that all 
%resources associated with the page are loaded through the proxy.

% and the domain of the proxy needs to 
%be inlined in all external requests. 




\paragraph{Security considerations}

A consequence of modifying the domain name is that the domain of the
target URL and the modified URL no longer matches, making them two 
separate origin as per the same-origin policy. This implies that all information in 
the browser specific to the target origin, e.g., cookies and local storage, 
are no longer associated with the modified origin, and vice versa. This results 
in a clean separation between the proxied and unproxied content.  

Another interesting consequence of altering the domain name, also relating to 
the same-origin policy, is the concept of domain relaxing. Modern web browsers 
allow relaxing of the same-origin policy for subdomains. Subdomains of the same 
domain can relax their domains by setting the \emph{document.domain} attribute
to their common domain. In doing so, they set aside the restrictions of the 
same-origin policy and can freely access each others resources across subdomains. 
Since the principle behind the suffix proxy is to make the target domain a subdomain of the proxy domain, two 
domains loaded via the proxy, each relaxing their domain to the domain of the 
proxy, can access each others resources across domains.
The consequence is that the suffix proxy has the opportunity to effectively disable the same-origin 
policy by adding a JavaScript snippet that sets the \emph{document.domain} property
to the proxy domain. This flexibility is beneficial to a monitor enforcing a policy that, e.g., is aiming 
at replacing the same-origin policy.
\todo{Daniel: yes, if expanded a bit}
\todo{Jonas: Does the last sentence make sense?}



\paragraph{Pros and cons}

While sharing a common foundation, there are several differences between a 
suffix proxy compared to a traditional web proxy. The differences lie, not in 
the way the monitor is included in the page, but in the way the proxy is 
addressed. One beneficial difference is that it does not require any 
configuration and leaves the browser completely unchanged. The user can take
advantage of the security benefits without being concerned with the integrity 
of the browser.

Another significant difference is that the suffix proxy lets the user decide which pages to proxy on a 
per-domain basis, making it more general than a traditional web proxy that covers 
all requests. %It can be seen as though a suffix proxy is off by default, whereas a web 
%proxy is on by default. 
Whether to apply the monitor is as simple as browsing to an alternate URL.


\todo{Integration stuff missing from intro? Or did I read too fast?}
\subsection{Integrator}

%The mashup integrator includes the monitor and decides which JavaScript should 
%run within the monitor. (Don't mention sandbox.) You're done.
%\begin{itemize}
%\item- No setup
%\item- Not intrusive to the browser
%\item- Not general
%\item- Deliberately intrusive to HTML
%\item- Integrator configuration, no user control
%\item- Run independent part of the code outside the monitor, possible performance gains
%\end{itemize}

As discussed earlier, today's
 web pages make extensive use of third-party code to add features 
and functionality to the page. The code is retrieved from external resources in 
the form of JavaScript libraries. The third-party code is considered to be part 
of the document and is executed in the same context as any other script 
included in the document. This gives the code full access to all the information 
of the page, including sensitive information such as form data and cookies. 
This requires that the code integrator must 
trust the library not to abuse this privilege. To a developer, an appealing alternative 
is to run untrusted code in the monitored context, while running trusted code outside of the 
monitor. 

\paragraph{Description}
Integrator-driven monitor inclusion is suitable for an integrating web page 
that makes use of third-party code.
% to extend the functionality to the page. 
The security of the information contained on the web 
page relies not only on the web page itself, but also on the security of all 
included libraries. To protect against malicious or compromised libraries, 
an integrator can execute part of, or all of the code in the monitor.
This can be achieved by manually including the monitor in the page and loading the 
third-party code through the suffix proxy. The suffix proxy will rewrite the 
response to direct execution of the library to the monitor. 

\paragraph{Security considerations}

\todo{Jonas: need help defining the security considerations}

\paragraph{Pros and cons}

This developer-centric approach gives the integrator full control over the 
configuration of the monitor and the policies to enforce. From the perspective of a user this 
approach is not intrusive to the browser, requires no setup or configuration, 
and provides additional security for the users sensitive information. However, 
it also limits the users control over which policies are applied to user information. 

Compared to the suffix proxy, integrator driven monitor 
inclusion is deliberately intrusive to the HTML code. Such specialized use of the suffix 
proxy allows a developer to specify precisely which code will be subject to 
monitoring, making the approach even more general. 
Since a monitor typically introduces a runtime overhead this approach can also 
result in performance gains compared to running all code monitored. Especially 
if the trusted code is computation heavy.



% 2-3 pages 
\section{Implementation}
\label{sec:impl}

%\todo{Daniel: may I suggest we replace the pros and cons with implementation challenges}
%Describe features and drawbacks with each implementation.
%For each deployment option, we describe how it is implemented and show how the integrity of the monitor
%and complete mediation are achieved.
This section details our implementations of the architectures
from the previous section. 

%%%%%%%%%%%%%%%%%%%%%%%%%%%%%%%%%%%%%%%%%%%%%%%%%%%%%%%%%%%%%%%%%%%%%%%%%%%%%%%%

\subsection{Browser extension}

The browser extension is a Firefox extension based on Zaphod~\cite{Zaphod}.
When loaded the extension first turns off the standard JavaScript engine by
disallowing JavaScript and listens for the DOMContentLoaded event.  The
DOMContentLoaded event is fired as soon as the DOM tree construction is
finished.  On this event the DOM tree is traversed twice. The first traversal
checks every node for event handlers, e.g., onclick, and registers the monitor
to handle them. The second traversal looks for JavaScript script nodes.  For
each found script node, the source is downloaded using XHR if needed, and the
monitor is used to execute the script.

As discussed above, the downside of this method is that it breaks
conservativeness and transparency, since the scripts all execute after the DOM
tree has been constructed. It is possible to regain transparency by using DOM
Mutation events instead of the DOMContentLoaded event.  The idea is to listen
to any addition of script nodes to the DOM tree under the construction, and
execute the script on addition.  However, due to performance reasons the DOM
Mutation events are deprecated, and are being replaced with DOM Mutation
Observers. It is unclear whether the DOM Mutation Observers can be used to
provide transparency, since events are grouped together, i.e., the mutation
observer will not necessarily get an event each time a script is added, but one
event covering all script additions.

However, the exact order of loading is not standardized and differs between
browsers. This forces scripts to be independent of such differences. Thus,
using the method of executing scripts on the DOMContentLoaded event is not
necessarily a problem in practice. 

%%%%%%%%%%%%%%%%%%%%%%%%%%%%%%%%%%%%%%%%%%%%%%%%%%%%%%%%%%%%%%%%%%%%%%%%%%%%%%%%

\subsection{Web proxy}

\begin{itemize}
\item Proxy implemented in NodeJS
%\item- rewrite all or nothing, problem with passing configuration to proxy
%\item DOM parsing to find scripts, compared to regular expressions
%\item difficult to identify all contexts (SVG, CSS and PDF can be rewritten, not flash)
%\item rewriting of JavaScript
\end{itemize}

The web proxy is implemented as a NodeJS~\cite{NodeJS} server. 

Wrapping JavaScript requires converting the source code to a string, which can 
be fed to the interpreting monitor. The source code is enclosed in quotes, and all 
quotes within the source code are escaped to not interfere with the enclosing quotes.
Additionally all escapes have to be escaped to not interfere with the escaped 
quotes. Next, because new-lines may change the semantics of the code due to 
JavaScript automatic semi-colon insertion, all new-line 
characters must be replaced with the JavaScript string equivalent.
These steps, exemplified in Listing~\ref{lst:wrap-monitor}, ensure that the wrapping cannot be evaded and that the source code 
will be semantically equivalent when interpreted by the monitor.

\begin{lstlisting}[label=lst:wrap-monitor, caption=Example of monitor wrapping]
code = 'Monitor.call("' + code.replace(/\\/g, '\\\\')
	.replace(/"/g, '\\"').replace(/\r?\n/g, "\\n\\\n")
	+ '")';
\end{lstlisting}
\todo{Jonas: fix listing}


Identifying inline JavaScript in HTML files is a complex task. 
Simple search and replace is not satisfactory due to the browser's error tolerant parsing of HTML-code, meaning that the 
browser will make a best-effort attempt to make sense of malformed fragments of 
HTML. It would require the search 
algorithm to account for all parser quirks in regard to malformed HTML;
a task which is at least as complex as actually parsing the document.
%As an example it is not regular expressions 
In the web proxy, Mozilla's JavaScript-based HTML-parser \emph{dom.js}~\cite{Mozilla:dom.js}, is used 
to parse the page. The DOM-tree can then be traversed to properly localize 
all inline script code. All occurences of JavaScript code are rewritten as 
outlined in Listing~\ref{lst:wrap-monitor}, wrapped in a call to the monitor.
Because all instances of the modified script code will reference the monitor, 
The monitor must be added as the first script to be executed.

\todo{Jonas: perhaps show a code snippet which would be difficult to search and replace? Example?}

\subsection{Suffix proxy (service)}

\begin{itemize}
%\item rewrite all or some, can pass configuration through cookie
%\item piggybacking, wildcard DNS
%\item SSL? wildcard certificates? features and drawbacks 
%\item disable SOP, domain relaxing?
%\item DOM parsing to find scripts, compared to regular expressions
%\item difficult to identify all contexts (SVG, CSS and PDF can be rewritten, not flash)
\item non-standard ports by forwarding all ports to one
\end{itemize}


\paragraph{Additional rewriting of HTML}


The web proxy provides a foundation for the suffix proxy, which is extended
with support for the wildcard subdomain approach and an additional rewriting step. 
To support wildcard subdomains, the proxy must strip the proxy domain to extract 
the target domain from the request. 
The additional rewriting is also related to the wildcard subdomain approach.
Since the suffix proxy is referenced by altering the domain name of the target, 
the proxy must ensure that resources associated with the target page are also 
retrieved through the proxy. Resources with URLs relative to the current domain 
requires no processing, as they are automatically loaded through the monitor. 
However, resources loaded from external domains must 
include the monitor domain. Similarly, links to external pages must include the domain of the monitor 
for the user to stay %within the 
monitored %domain 
when links are followed.
Apart from the rewriting of inline 
scripts and event handlers done in the web proxy, the suffix proxy must 
identify and rewrite the domain of all such external references. 


Recall from the previous section the rather complex procedure required to add HTTPS support.
Implementing support for HTTPS is easier in the suffix proxy compared to the web proxy. 
Given that the suffix proxy builds on DNS wildcards, it is sufficient to issue a certificate
for all subdomains of the proxy domain, e.g., \emph{*.proxy.domain}. Such a 
certificate is valid for all target URLs relayed through the proxy.
The main benefit of wildcard certificates is that the user can browse HTTPS pages. 
The drawback is that the user must trust the monitor to correctly verify the 
certificate of the page.

Another difference to the web proxy relates to the use of non-standard ports.
The web proxy will receive all requests regardless of the target port. The 
suffix proxy, on the other hand, only listens to the standard ports for HTTP and HTTPS, port 80 and 
443 respectively. Since the port in a URL is specified in conjunction to, but not included in the domain.
Hence any URLs specifying non-standard ports would attempt to connect to closed 
ports on the proxy server. The solution to this problem is to locally redirect 
all external ports to the standard ports. This can be done using \emph{iptables} in Linux, 
Listing~\ref{lst:iptables} shows an example of the command to setup iptables to do this.

\begin{lstlisting}[label=lst:iptables, caption=Example of redirecting ports using iptables]
iptables -t nat -A PREROUTING -p tcp --dport 1:65535 -j REDIRECT --to-ports 80
\end{lstlisting}
\todo{Jonas: fix listing}

\subsection{Integrator}

\begin{itemize}
\item- rewrite some, integrator driven, no or limited user configuration
\item- no need for parsing, integrator decides
\item- well defined per site policy
\item- site could allow user to configure policy
\item- requires developer understanding of the monitor
\item- future reflection of monitored code?
\end{itemize}

%%%%%%%%%%%%%%%%%%%%%%%%%%%%%%%%%%%%%%%%%%%%%%%%%%%%%%%%%%%%%%%%%%%%%%%%%%%%%%%%

% 2-3 pages 
\section{Instantiation}
\label{sec:case}

This section presents practical experiments made by instantiating the deployment
methods with the JSFlow~\cite{Hedin:Sabelfeld:CSF12,JSFlow} information-flow monitor. 

\subsection{Monitor}

JSFlow is a dynamic information-flow monitor by tagging values with runtime
security labels used for security decisions. Whenever a potential security
violation has been encountered the monitor stops the execution with a security
error. Suiting our need, the monitor implemented in JavaScript and supports
full non-strict Ecma-262 (v5) including the standard API and large parts of the
browser specific execution environment including the DOM.  From an information
flow perspective, JSFlow supports a wide variety if information flow policies,
including tracking of user input preventing it from leaving the browser used in
the security experiments described below.

\todo{The second site causes security errors due to NSU; shall we simply not agree on releasing the code if someone asks? Or shall we discuss the matter here?}

\subsection{Experiment scenarios}

Consider the scenario of a password strength checker. The use the service
you input a password and the strength of the password is computed according
to some metric and the result is displayed to the user, typically on a scale
from \emph{weak} to \emph{strong}.
%
For this kind of service confidentiality of the password is of paramount importance --- 
the strength of the password is of no importance if the password is not a secret.
Unless the method of computing strength is a trade secret the service can be
implemented as a client side service, and the password does not have to leave the
browser. Sending back the password to the service provider to perform
the strength computation puts a lot of trust in the service provider to transfer the
password in a secure way, and make sure that no information about the password
is stored. For vital secrets such as passwords, there is no reason for the user
to give this trust. Hence, in this situation, it is important to guarantee that
the password does not leave the browser.

We have investigated a number of password strength services. The services can
be categorized based on whether the computation is server side or client side.
Below we report on our results with two services selected from each category. 
Even though it is conceivable to guarantee security of the client side service
by isolation the service uses Google Analytics, which requires the ability
to communicate over the Internet to function. In general it is not possible
to achieve security via isolation without impairing the functionality of
the service.

\todo{Way of referecing? use citations?}
The first is the insecure site \url{https://testalosenord.pts.se/} provided by
the Swedish authority PTS. The service presents the user with an input field
and a button to initiate the test. Once the button is pressed, the
password is sent to the service provider.
\todo{Note: it is over https so it is not in clear text}

The second is the service
\url{http://www.getsecurepassword.com/CheckPassword.aspx}, which provides a
secure client side password strength checker. The user is presented with an
input box, and the strength of the password is computed as the password
is typed in.
%
It is interesting to note that even if the second service is secure, the input
field used to input the password is contained in a form. Since the input field
is not given a name, the password is not included when the form is posted.
However, the small change of adding a name to the field would cause the
password to be sent back in cleartext to the service provided rendering the
service insecure.

\paragraph{Browser extension}
%
Loading a page in Firefox with the extension enables causes the monitor to be used
as the execution engine. In addition, the extension captures important events that
might cause information to leave the browser, e.g., \lstinline{onsubmit}. 

Loading the first site presents the user with the input form and the button. The user
is allowed to type in the password, but as soon as the user presses the button to perform
the check the monitor detects that the password is sent to the service provider
as part of the post data and stops execution with a security error.

Loading the second site presents the user with the input form and the strength of the password 
is gradually computed as it is typed. If the user presses enter the form is posted back to
the service provider, but since the password is not part of the post data, execution is
allowed to continue. If, however, a name is added to the input field, the password is part
of the post data, and the monitor stops execution with an error.


\paragraph{Web proxy}

Loading a page through the inlining web proxy causes all scripts to be wrapped
with the monitor. As above, the monitor is also registered as the handler to
important events that might cause information flow. The result is the same
for the two sites. Attempting to compute the password strength using the first
site causes execution to be stopped with a security error, whereas the second
site is allowed to run.

\paragraph{Suffix proxy}

Loading the page via the suffix proxy causes the monitor to be injected into the page in 
a similar way to the web proxy, with identical result.

\subsection{Performance experiments}
%Jonas

%1 page 
\section{Related work}
\label{sec:related}
We first discuss original work on reference monitors and their
inlining, then
inlining security checks in the context of
JavaScript, to relate to the architectures, 
and finally inlining for information-flow, to relate to our instantiations
of the architectures.

\paragraph{Inlined reference monitors}
Anderson~\cite{Anderson:72} introduces reference monitors and
outlines the principle requirements, recounted in Section~\ref{sec:intro}.
%
Erlingsson and Schneider~\cite{DBLP:conf/nspw/ErlingssonS99,Erlingsson:PhD04} instigate
the area of inlining reference monitors. 
This work studies both enforcement mechanisms and the policies
that they are capable of enforcing, with the focus on safety properties.
Inlined reference monitors
have been proposed in a variety of languages and settings: from
assembly code~\cite{DBLP:conf/nspw/ErlingssonS99} to Java~\cite{DBLP:conf/ecoop/DamJLP09,DBLP:journals/jcs/DamJLP10,DBLP:conf/ccs/DamGL12}.

Ligatti et al. \cite{Ligatti05editautomata:} present a 
general framework for security policies that can
be enforced by monitoring and modifying programs at runtime. 
They introduce \emph{edit automata} that enable
monitors to stop, suppress, and modify the behavior of programs. 



\paragraph{Inlining for secure information flow}
Language-based information-flow security~\cite{Sabelfeld:Myers:JSAC}
features work on inlining monitor for security information flow.
Secure information flow is not
a safety property~\cite{McLean:SSP94}, but can be approximated by
safety properties
(e.g.,~\cite{Boudol:FAST08,Sabelfeld:Russo:PSI09,Austin:Flanagan:PLAS09}).

Chudnov and
Naumann~\cite{Chudnov:Naumann:CSF10} have investigated an inlining
approach to monitoring information flow in a simple imperative language. They inline a flow-sensitive
hybrid monitor by Russo and
Sabelfeld~\cite{Russo:Sabelfeld:CSF10}. The soundness of the inlined
monitor is ensured by bisimulation of the inlined monitor and the
original monitor.

Magazinius et
al.~\cite{Magazinius+:SEC10,DBLP:journals/compsec/MagaziniusRS12} show
how to cope with dynamic code evaluation instructions by inlining
on-the-fly. 
Dynamic code evaluation instructions are
rewritten to make use of auxiliary functions that, when invoked at
runtime, inject security checks into the available string. 
The inlined code manipulates shadow variables to keep track of the
security labels of the program's variables.

However, there are fundamental limits in the scalability of the shadow-variable
approach.  The execution of a vast majority of the JavaScript operations (with
the prime example being the \lstinline{+} operation) is dependent on the types
of their parameters.  This might lead to coercions of the parameters that, in
turn, may invoke \lstinline{toString} and \lstinline{valueOf}. In order to take
any side effects of those methods into account any operation that may case
coercions must be wrapped. The end result of this is that the inlined code ends
up emulating the interpreter, leaving no advantages to the shadow-variable
approach.

\paragraph{Inlining for secure JavaScript}
Inlining has been explored JavaScript, although often focusing on
simple properties or preventing against fixed classes of vulnerabilities.
A prominent example in the context of the web is
BrowserShield~\cite{Reis+:TWeb07} by Reis et al. to instrument scripts with
checks for known vulnerabilities.

Yu et al.~\cite{Yu+:POPL07} and Kikuchi et
al.~\cite{DBLP:conf/aplas/2008} present an instrumentation approach
for JavaScript in the browser. Their framework allows instrumented
code to encode edit automata-based policies.

Phung et al.~\cite{DBLP:conf/ccs/PhungSC09} and 
Magazinius et al.~\cite{DBLP:conf/nordsec/MagaziniusPS10} develop
secure wrapping for self-protecting JavaScript. This approach is based
on wrapping built-in JavaScript methods with secure wrappers that
keep track of the security state and regulate access to the original built-ins.  



Agten et al.~\cite{DBLP:conf/acsac/AgtenABPDP12} present JSand, a
server-driven client-side sandboxing framework. The framework mediates
attempts of untrusted code to access to resources in the browser.  
In contrast to its predecessors such as
ConScript~\cite{DBLP:conf/sp/MeyerovichL10}, WebJail~\cite{DBLP:conf/acsac/AckerRDPJ11}, and Contego~\cite{DBLP:conf/trust/LuoD11},
the sandboxing is done purely at JavaScript level, requiring no
browser modification.

Despite the above progress on inlining security checks in JavaScript, achieving
Information-flow security for client-side JavaScript by inlining has been out of
reach for the current methods~\cite{Vogt+:NDSS07,DBLP:conf/pldi/ChughMJL09,Yip:Narula:Krohn:Morris:EUROSYS09,Jang+:CCS10,DeGroef+:CCS12}  that either
modify the browser or perform the analysis out-of-the-browser.

%1 page 
\section{Conclusions}
\label{sec:conc}
Different stakeholders have different interests in the security of web
applications. We have presented architectures for inlining security
monitors, to take into account the security goals of the users, system and
network administrators, and service providers and integrators.
%
We achieve great flexibility in the deployment options by considering
security monitors implemented as security-enhanced JavaScript interpreters.
%
The architectures allow deploying such a monitor in a browser
extension, web proxy, or web service.
%
We have reported on the security considerations and on the relative pros and
cons for each architecture.
%
We have applied the architectures to inline an information-flow
security monitor for JavaScript.
%
The security experiments show the
flexibility in supporting the different policies on the sensitive
information from the user. 
%
The performance experiments show reasonable overhead imposed by the
architectures and point out to the relative pros and cons from the
performance point of view. \todo{Make a bit more concrete when we have
the experimental results.}

Future work \todo{Daniel, Jonas: what's worth brining up here? Some
tracks that give us the credibility - pointing out a rich research agenda, rather
than pointing out the weaknesses of the current paper. :) }

\paragraph{Acknowledgments}
This work was funded by 
the European Community under the ProSecuToR and WebSand projects
and
the Swedish research agencies SSF and VR.


% Less than 4 pages 
\bibliographystyle{plain}
\bibliography{literature}
\end{document}
