\documentclass{llncs}
%\documentclass[10pt, conference, compsocconf]{IEEEtran}

% From ESORICS'13 CFP
% Submitted papers should be at most 16 pages (using 10-point font), excluding 
% the bibliography and well-marked appendices, and at most 20 pages total.

\usepackage{listings}
\usepackage{url}

\usepackage{amssymb}
\usepackage[usenames]{color}
\definecolor{lightred}{rgb}{1,0.8,0.8}
\newcommand{\todo}[1]{\colorbox{red}{\textcolor{white}{\sffamily\bfseries\scriptsize TODO}} \textcolor{red}{#1} \textcolor{red}{$\blacktriangleleft$}}


\title{Architectures for Inlining Security Monitors in Web Applications}

\author{Jonas Magazinius \and Daniel Hedin \and Andrei Sabelfeld}
\institute{Chalmers University of Technology, Gothenburg, Sweden}
%%%%%%%%%%%%%%%% Schedule %%%%%%%%%%%%%%%%%%%%%%%%
\if 0
March 28 - paper shipped
March 27 - final polish
March 22 - intro and conclusions
March 20 - case studies + related work
March 17 - architecture and implementation done 
March 12 - implementation half-way
March 11 - architecture
\fi
%%%%%%%%%%%%%%%%%%%%%%%%%%%%%%%%%%%%%%%%%%%%%%%

\begin{document}


\maketitle



\begin{abstract}
Securing JavaScript in the browser is an open and challenging
problem. Code from pervasive third-party JavaScript libraries exacerbates the
problem because it is executed with the same privileges as the code
that uses the libraries.
%
An additional complication is that the different stakeholders have
different interests in the security policies to be enforced
in web applications.
%
This paper focuses on securing JavaScript code by \emph{inlining}
security checks in the code before it is executed.
%
We achieve great flexibility in the deployment options by considering
security monitors implemented as security-enhanced JavaScript interpreters.
%
%A key ingredient to open up for such flexibility is security monitoring in the form of a
%security monitor for JavaScript, written itself in JavaScript.
We propose three architectures for inlining security monitors for JavaScript: in
the browser, as a web, and as a suffix proxy (web service). 
%
These architectures offer great flexibility: they are parametric in the monitor itself,
and they provide freedom in the choice of where the monitor is
injected, allowing to serve the interests of the different stake
holders: the users, code developers, code
integrators, as well as the system and network administrators.
%
We report on experiments that demonstrate successful deployment of a JavaScript
information-flow monitor with all three architectures.
\end{abstract}


% \begin{IEEEkeywords}
% Web Security; Polyglot; Injection; Cross-domain.
% \end{IEEEkeywords}








% 1-2 pages
\section{Introduction}
\label{sec:intro}

\todo{Daniel: I assume that we give a description on IF and monitors for IF in general in this section}

When introducing reference monitoring, Anderson~\cite{Anderson:72}
identifies the following requirements:
\begin{itemize}
\item \emph{monitor integrity}: the monitor must be tamperproof,
\item \emph{complete mediation}: the monitor must be always invoked, and
\item \emph{soundness}: the monitor must enforce the desired
  security policy.
\end{itemize}
In addition, requirements of \emph{conservativeness} and
\emph{transparency} are often in place for reference monitors to
ensure that no new behaviors are added or removed by monitors when the
original program is secure in the first place.

This paper proceeds as follows. Section~



\paragraph{Shadowing}
\todo{Daniel: me must make sure to differentiate between the inlining ala older papers and the inlining we do here.}

Inline the monitor into the JavaScript code. Appealing idea, previous work for 
toy languages, but difficult in practice. Abandoned due to complexity with 
coercion in expressions.


A growing number of web pages provide not only content, but services to their
users. In order to provide the service the provider requires the user to
provide potentially sensitive information such as user credentials, and payment
information. For such pages it is in the interest of the user that the
sensitive information is only disclosed to the intended recipient.  Today, no
browsers offer such guarantees. However, modern browsers allow the
functionality of the browser to be enriched via \emph{extensions}. 


\paragraph{Third-party code integration}
%\label{sec:mash}

Describe mashups, difficulties in mashup security and information flows in mashups.

%4 pages 
\section{Architecture}
\label{sec:arch}

This section presents the different architectures for inlining
security monitors. We motivate each architecture by scenarios of
intended usage, describe the details of the architectures, report on
security considerations as well as the pros and cons for each choice.



%%%%%%%%%%%%%%%%%%%%%%%%%%%%%%%%%%%%%%%%%%%%%%%%%%%%%%%%%%%%%%%%%%%%%%%%%%%%%%%%

\subsection{Browser extension}

Many websites require the user to provide sensitive information, e.g., user
credentials, or payment information. It lies in the interest of the user to
ensure the security of the sensitive information.  Modern browsers allow for
the functionality of the browser to be enriched via \emph{extensions}. By
deploying the security monitor via a browser extension it is possible to enforce
properties not normally offered by browsers.  Regardless of whether the user is
private or corporate, browser extensions provide a simple 
install-once-protect-always deployment method.


\paragraph{Description}

A browser extension is a program that is installed into the browser in order to
change or enrich the functionality of the browser.  The basic idea behind
deploying via an extension is to replace the JavaScript engine with the monitor.
(Recall that we consider only monitors that function as interpreters.)
%
This can be achieved by turning off the standard JavaScript engine, and have
the extension traverse the page when loaded and execute the JavaScript scripts
using the monitor. 
%
The most common extension implementation language is JavaScript. Given that
the monitor is written in JavaScript it is a simple matter to embed the 
monitor in an extension.
This method was pioneered by Zaphod~\cite{Zaphod}, a
FireFox extension, that replaces the standard JavaScript engine with the
experimental Narcissus engine.

\paragraph{Security considerations}

The integrity of the method is guaranteed by virtue of the fact that the
JavaScript engine is turned off and all scripts are interpreted by the monitor.
Hence, the scripts are not active; rather, they are passed as data to the
monitor, and are only able to influence the execution environment implemented
by the monitor and not the general execution environment of the monitor itself.
Since all scripts are executed using the monitor complete mediation is implied.
In addition, this also implies that the deployment method is sound given that
the monitor is sound. 

As described above browser extension are installed into the browser. In order
to be able to enrich the functionality of the browser the extensions run with
the same privileges as the browser. This entails that the monitor will be
running with elevated privileges. Compared to the other methods of deployment
this means that a faulty monitor not only jeopardizes the property enforced by
the monitor, but might jeopardize the integrity of the entire browser.

\paragraph{Pros and cons}
\todo{Andrei: suggestion for uniform structure across the different
  sections - start by the pros and
  then present the cons ensuring that there is no less text for the
  pros than for the cons :)}
\todo{discuss}
The main benefit of the approach is that the extension gives the user direct
control over what security policies to enforce on the browsed pages. For
instance, it is possible to have any security violations reported to the user,
who can decide to allow or decline the violation. 

Whether the requirements of conservativeness and transparency can be met
depends on when the scripts are executed and that all functionality provided by
the browser can be implemented by the monitor.

Consider first the 
order in which scripts are executed.
%Andrei: this might be misinterpreted as the time it takes for scripts to execute 
% execution time of scripts
When web pages are loaded, the
scripts of the pages are executed as they are encountered while parsing the web
page. This means that the DOM tree of the page might not have been fully
constructed when the scripts execute.  For scripts that interact with the DOM
tree this is important, in particular if the script injects nodes into the
tree. From a conservativeness and transparency perspective it is not entirely
satisfactory to traverse the DOM tree of the page after it has been
constructed.  Rather, scripts need to be executed as soon as they are added to
the tree, which can be achieved using DOM mutation events.

Another obstacle for \todo{conservativeness and?} transparency is that not all
functionality provided by the browser can be reproduced by the monitor. One
example of this is \emph{document.write}. The semantics of
\emph{document.write} is loosely specified~\todo{reference} saying that \emph{document.write}
writes a string into the current position of the document.  Intuitively,
\emph{document.write} writes into the character stream that is fed to the HTML
parser, which, for obvious reason can have drastic effects on the parsing of
the page. For security reasons extensions are prohibited from calling
\emph{document.write}. Hence, when run via an extension the monitor cannot
easily provide the full functionality of \emph{document.write}.
\todo{explain what can be provided easily and what is the difficult
  way to provide full functionality if any}
\todo{example, ala div}

On the positive side features like \emph{innerHTML} can be handled with
relative ease. \emph{innerHTML} is a property on all DOM nodes which allows the
text representation of the node to be changed. Any change of the
\emph{innerHTML} property causes the content of the node to be changed by
parsing the changed property. For other deployment methods this means that the
value written to \emph{innerHTML} must be validated, and possibly changed
before the write is allowed to occur --- even though script nodes added
using \emph{innerHTML} are not executed, event handlers might still be run.
%
For the extension based deployment this
is not a problem.  Since the standard JavaScript execution engine is turned off
it is safe to write to the \emph{innerHTML} property. The write will trigger
the parser to rebuild the content of the node, but no scripts will be executed, 
and the result can be handled in a manner analogous to the scripts on
a loading page.




Finally, the extension is entirely independent of the method of transport,
since the extension can tell the browser to download the resource using XHR.
Thus, unlike other methods of deployment it is unimportant whether the scripts
are fetched using HTTP or HTTPS.

%%%%%%%%%%%%%%%%%%%%%%%%%%%%%%%%%%%%%%%%%%%%%%%%%%%%%%%%%%%%%%%%%%%%%%%%%%%%%%%%

\subsection{Web proxy}

In a corporate setting, it is essential for a company to protect 
sensitive information pertaining to the company, as well as their employees. 
The common use of intranet portals to facilitate easy sharing of information 
within the company puts this sensitive information in the web browser. 
Meanwhile, employees are using the same web browsers for accessing untrusted content on 
the Internet. Naturally, it is in the interest of the company to ensure that 
the sensitive information is not accessible outside the domain of the company.
In this distributed setting a \emph{web proxy} is a convenient way of delivering 
the security monitor to all clients with minimal effort. 

This is also appealing to a private user, as it includes the monitor in all browsed pages without modifying 
the browser itself. By configuring the browser to use a web proxy, the user can ensure the security 
of sensitive data.

\paragraph{Description}
A proxy specific to relaying HTTP requests is often referred to as a web proxy.
All modern browsers support using web proxies to relay requests.
It allows all outgoing HTTP requests to be 
relayed through the proxy server. The web proxy will act as a middle-man, making 
requests on behalf of the client. In the process, the proxy has the 
opportunity of modifying both the request and the response. 
As a monitor deployment method the opportunity to rewrite the response is 
of interest. A web proxy is a convenient way to include the monitor in browsed pages without 
modifying the browser itself. The browser is configured to relay 
all requests through the proxy, and the proxy rewrites the responses to include 
the monitor in each page. Modifying the response makes the method more 
intrusive to the HTML content, but less intrusive to the users browser. 

For the monitor to have any effect, all scripts bundled with 
the page must be executed within the monitor. The scripts can either be inline, 
e.g. included as part of the HTML page, or external, e.g. referenced in the 
HTML page to be downloaded from an external source.
Apart from including the monitor in all browsed pages, all inline and external 
scripts must be identified and rewritten by the web proxy to direct execution to the monitor.
This is achieved by wrapping all JavaScript source code, whether inline or external, in a call to the monitor.
In the web proxy different rewriting rules apply depending on whether the 
requested content is JavaScript, e.g. external, or HTML, e.g. inline. External 
JavaScript content is simply wrapped, whereas 
for HTML the web proxy has to identify and wrap all occurrences of inline 
JavaScript, both script-tags and event handlers.

Wrapping JavaScript requires converting the source code to a string, which can 
be fed to the interpreting monitor. The source code is enclosed in quotes, and all 
quotes within the source code is escaped to not interfere with the enclosing quotes.
Additionally all escapes have to be escaped to not interfere with the escaped 
quotes. Next, because new-lines may change the semantics of the code due to 
JavaScript automatic semi-colon insertion, all new-line 
characters must be replaced with the JavaScript string equivalent.
These steps ensure that the wrapping cannot be evaded and that the source code 
will be semantically equivalent when interpreted by the monitor.

\begin{lstlisting}{}
'Monitor.call("' + code.replace(/\\/g, '\\\\').replace(/"/g, '\\"').replace(/\r?\n/g, "\\n\\\n") + '")';
\end{lstlisting}
\todo{Jonas: fix listing}

Identifying inline JavaScript in HTML files is a complex task. 
Due to the browser's error tolerant parsing of HTML-code, meaning that the 
browser will make a best-effort attempt to make sense of malformed fragments of 
HTML, simple search and replace is not satisfactory. It would require the search 
algorithm to account for all parser quirks in regard to malformed HTML.
A task which is at least as complex as actually parsing the document.
%As an example it is not regular expressions 
In the web proxy, a HTML-parser is utilized 
to parse the page. The parse-tree can then be traversed to properly localize 
all inline script code. All occurences of JavaScript code is rewritten as 
outlined above, wrapped in a call to the monitor.
Because all instances of the modified script code will reference the monitor, 
The monitor must be added as the first script to be executed.

\todo{Jonas: perhaps show a code snippet which would be difficult to search and replace? Example?}


\paragraph{Security considerations}

As opposed to a browser extension, which replaces the 
JavaScript engine, the monitor will be executed by the engine of the browser. 
By including the monitor in the page, it is executed in the context of the page. Naturally 
this is the same execution context in which all scripts bundled with the 
page are executed. Whereas a 
browser extension intercepts execution of all scripts by replacing the JavaScript 
engine of the browser, a web proxy must identify all inline and 
external scripts and rewrite them to ensure that they are executed within the 
monitored execution context. Failing to do so, script code will execute 
along-side the monitor, giving it access to the sensitive information the 
monitor was intended to protect. 
This effort continues as the monitor executes. Various JavaScript features, such as 
\emph{document.write}, allow an arbitrary string to be 
interpreted and rendered as HTML. Any scripts present in the string will be 
executed upon interpretation. The monitor must account for this and again rewrite 
the string to direct execution to the monitor.

Special consideration is required for secure HTTP connections (HTTPS). In an HTTPS 
connection, a certificate containing the public key of the target domain is sent 
along with the response. The response is encrypted using the private key of the target 
domain and decrypted with the public key contained in the certificate. The 
certificate is issued by a certificate authority (CA)
that validates the certificate by signing the public key. Given that the user 
trusts the CA, it can also trust that the public key belongs 
to the target domain, and that the response has not been modified in transit.
This poses a challenge for a web proxy designed to exactly that. The solution 
is for the proxy to act as a CA by generating a
new certificate for the target domain on-the-fly, and sign the 
public key of the certificate with the private key of the proxy. Naturally the 
the proxy must be trusted by the user, either because the key is added to 
the users list of trusted CAs, or signed by one that the user trusts.

Complete mediation is achieved since all requests are relayed through the proxy 
and all occurrences is wrapped to direct execution to the monitor.

\todo{Jonas: Need to expand.}

\paragraph{Pros and cons}

In a corporate environment there are several benefits of using a web proxy 
compared to a browser extension. As an example, a web proxy can be centrally 
administered and configured, whereas a browser extension requires individual installation 
and configuration on each computer. Hence, using this approach can significantly 
reduce administration efforts in a larger corporation. 
Another benefit of using a web proxy is that it is browser independent. The 
rewriting technique is the same regardless of the browser used, hence, there is no 
need to adapt the web proxy to suit a particular browser, allowing 
users their choice of browser.
Neither is it intrusive to the browser, as the user is not required to modify the 
browser in order to take advantage of security benefits. 
However, there 
are certain drawbacks as well. 

In order for the monitor to be effective, the web proxy 
has to identify all scripts. The process of doing so may require additional 
resources and introduce additional 
latency in comparison to that of a usual web proxy.

As an individual user, the main benefits are browser independence and  
increased security without having to modify the browser. How the proxy is 
deployed determines the users influence over 
the policies enforced. If deployed locally, the user has full influence over 
how policies are to be used, while for a remotely deployed proxy, the user is 
in the hand of the proxy administrator. 

Typically a web proxy relays and includes the monitor in all browsed pages, 
trusted or not, will run monitored. With a browser extension a user can
configure certain trusted web pages to run unmonitored for performance reasons.


\subsection{Suffix proxy (service)}
%\todo{Jonas: Just found out that the "service" we developed is actually a suffix proxy, which actually could have been implemented in a simpler way using .htaccess request rewriting. Also found a way to redirect all ports to a single port using iptables. Perfect solution.}
%\begin{itemize}
%\item What is a web proxy
%\item How can a web proxy help in distributing the monitor
%\item 
%\item Delivering the associated cookies?
%\item Same-origin policy in respect to different design choices
%\item More intrusive in HTML content, rewrite links and external references 
%\item Browser independent? Depends on the monitor
%\item Not intrusive to the browser
%\item Even less cumbersome to setup, browse to server URL
%\item Server-side and client-side configuration possible?
%\item Less general, only browsed pages
%\end{itemize}

\todo{Users is one perspective, but  service providers is another
  important one}
Certain users want to distinguish between trusted and untrusted web sites.
A \emph{suffix proxy} particularly suits a user who wants to run certain untrusted 
pages monitored to prevent leakage.

\paragraph{Description}
A \emph{suffix proxy} is a specialized web proxy, with a different approach to relaying the request. 
Instead of the common practice of configuring the browser to relay all requests through a web proxy,
the suffix proxy makes use of DNS wildcards to redirect the request to it.
Typically, the user navigates to a 
web application associated with the proxy and enters the target URL, e.g. \emph{http://google.com/search?q=proxy}, in 
an input field. To direct the request to the proxy the target domain name is altered
by appending the domain name of the proxy, making the target domain a sub-domain of the proxy domain, e.g. \emph{http://google.com.proxy.domain/search?q=proxy}. 
The browser is then navigated to the modified URL.
The suffix proxy is set up such
that all requests to any sub-domain of the proxy domain are directed to the proxy domain, 
e.g. in DNS terms \emph{*.proxy.domain $\Rightarrow$ proxy.domain}. 
A web application on the proxy domain is set up to listen for such sub-domain requests.
When a request for a sub-domain is registered, it is intercepted by the web application.
The web application strips the proxy domain from the URL, leaving the original target URL, 
and makes the request on behalf of the client. As in the case of the web proxy, 
relaying the request to the target URL gives the suffix proxy an opportunity to modify and 
include the monitor in the response.

A consequence of modifying the domain name is that the domain of the
target URL and the modified URL no longer matches, making them two 
separate origin as per the same-origin policy. This implies that all information in 
the browser specific to the target origin, e.g. cookies and local storage, 
are no longer associated with the modified origin, and vice versa. This results 
in a clean separation between the proxied and unproxied content.  

Since the suffix proxy is referenced by altering the domain name of the target, 
the proxy ensures that resources associated with the target page is also 
retrieved through the proxy. Resources with URLs relative to the current domain 
requires no processing, as they are automatically loaded through the monitor. 
However, resources loaded from external domains must 
include the monitor domain. Similarly, links to external pages must include the domain of the monitor 
for the user to stay %within the 
monitored %domain 
when links are followed.
Apart from the rewriting of inline 
scripts and event handlers done in the web proxy, the suffix proxy must 
identify and rewrite the domain of all such external references. 

%The latter implies that apart from rewriting scripts within a page, all URLs to 
%external resources, i.e., to origins outside the scope of the original origin, 
%must be rewritten to include the domain of the proxy. This ensures that all 
%resources associated with the page are loaded through the proxy.

% and the domain of the proxy needs to 
%be inlined in all external requests. 


Recall from the previous section the rather complex procedure required to add HTTPS support.
Implementing support for HTTPS is easier in the suffix proxy compared to the web proxy. 
Given that the suffix proxy builds on DNS wildcards, it is sufficient to issue a certificate
for the all subdomains of the proxy domain, e.g. \emph{*.proxy.domain}. Such a 
certificate is valid for all target URLs relayed through the proxy.


\paragraph{Security considerations}

Appending the domain name of the proxy allows the browser to relax the domain of the page to that of the proxy, allowing it to communicate freely with any other proxied page doing the same.

Similarly to a web proxy, all inline and external script must be rewritten, but to accomplish this, the suffix proxy must also rewrite all external references to go through the monitor.

\paragraph{Pros and cons}




There are several differences between a suffix proxy compared to a traditional 
web proxy. One beneficial difference is that it does not require any configuration and leaves the browser 
completely unchanged. Another difference is that the suffix proxy lets the user decide which pages to proxy on a 
per-domain basis, making it more general than a traditional proxy that covers 
all requests. 


\subsection{Integrator}

%The mashup integrator includes the monitor and decides which JavaScript should 
%run within the monitor. (Don't mention sandbox.) You're done.
%\begin{itemize}
%\item- No setup
%\item- Not intrusive to the browser
%\item- Not general
%\item- Deliberately intrusive to HTML
%\item- Integrator configuration, no user control
%\item- Run independent part of the code outside the monitor, possible performance gains
%\end{itemize}

Today web pages make extensive use of libraries to extend the functionality of 
the web page. As the code is executed in the context of the web page, it can 
access the content of the page. This requires that the code integrator must 
trust the library not to abuse this privilege. To a developer, an appealing alternative 
is to run untrusted code monitored, while running trusted code outside of the 
monitor. This can be achieved including the monitor in the page and loading the 
third-party code through the suffix proxy. The suffix proxy will rewrite the 
response to direct execution of the library to the monitor. This can also result in performance 
gains compared to running all code monitored, since a monitor typically introduces 
a runtime overhead. Compared to the suffix proxy, integrator driven monitor 
inclusion is deliberately intrusive to the HTML code. Such specialized use of the suffix 
proxy allows a developer to specify precisely which code will be subject to 
monitoring, making the approach even more general. 

This developer centric approach gives the developer full control over the 
configuration of the monitor and the policies to enforce. To a user this 
approach is not intrusive to the browser, requires no setup or configuration, 
but also limits the users control. 


% 2-3 pages 
\section{Implementation}
\label{sec:impl}

%\todo{Daniel: may I suggest we replace the pros and cons with implementation challenges}
%Describe features and drawbacks with each implementation.
%For each deployment option, we describe how it is implemented and show how the integrity of the monitor
%and complete mediation are achieved.
This section details our implementations of the three architectures
from the previous section. 

%%%%%%%%%%%%%%%%%%%%%%%%%%%%%%%%%%%%%%%%%%%%%%%%%%%%%%%%%%%%%%%%%%%%%%%%%%%%%%%%

\subsection{Browser extension}

The browser extension is a Firefox extension based on Zaphod~\cite{Zaphod}.
When loaded the extension first turns off the standard JavaScript engine by
disallowing JavaScript and listens for the DOMContentLoaded event.  The
DOMContentLoaded event is fired as soon as the DOM tree construction is
finished.  On this event the DOM tree is traversed twice. The first traversal
checks every node for event handlers, e.g., onclick, and registers the monitor
to handle them. The second traversal looks for JavaScript script nodes.  For
each found script node, the source is downloaded using XHR if needed, and the
monitor is used to execute the script.

As discussed above, the downside of this method is that it breaks transparency,
since the scripts all execute after the DOM tree has been constructed. It is
possible to regain transparency by using DOM Mutation events instead of the
DOMContentLoaded event.  The idea is to listen to any addition of script nodes
to the DOM tree under the construction, and execute the script on addition.
However, due to performance reasons the DOM Mutation events are deprecated, and
are being replaced with DOM Mutation Observers. It is unclear whether the DOM
Mutation Observers can be used to provide transparency, since events are
grouped together, i.e., the mutation observer will not necessarily get an event
each time a script is added, but one event covering all script additions.
\todo{Andrei: Inconclusive ending - please finish on a punchy and positive
  note! :)}

%%%%%%%%%%%%%%%%%%%%%%%%%%%%%%%%%%%%%%%%%%%%%%%%%%%%%%%%%%%%%%%%%%%%%%%%%%%%%%%%

\subsection{Web proxy}

\begin{itemize}
\item Proxy implemented in NodeJS
%\item- rewrite all or nothing, problem with passing configuration to proxy
\item DOM parsing to find scripts, compared to regular expressions
\item difficult to identify all contexts (SVG, CSS and PDF can be rewritten, not flash)
\item rewriting of JavaScript
\end{itemize}



\subsection{Suffix proxy (service)}

\begin{itemize}
%\item rewrite all or some, can pass configuration through cookie
\item piggybacking, wildcard DNS
\item SSL? wildcard certificates? features and drawbacks 
\item disable SOP, domain relaxing?
\item DOM parsing to find scripts, compared to regular expressions
\item difficult to identify all contexts (SVG, CSS and PDF can be rewritten, not flash)
\item non-standard ports by forwarding all ports to one
\end{itemize}


\paragraph{Additional rewriting of HTML}


The web proxy was implemented in Node.js. 

There are different technical approaches to communicate the target URL to the web 
proxy. The URL can either be sent as a parameter of the request, 

The features of wildcard certificates is that the user can browse HTTPS pages. The drawback is that the user must fully trust that the monitor correctly verifies the signature of the page.

\todo{same origin policy between pages from different original domains loaded through the monitor can be relaxed to the origin of the proxy and share information freely}



\subsection{Integrator}

\begin{itemize}
\item- rewrite some, integrator driven, no or limited user configuration
\item- no need for parsing, integrator decides
\item- well defined per site policy
\item- site could allow user to configure policy
\item- requires developer understanding of the monitor
\item- future reflection of monitored code?
\end{itemize}

%%%%%%%%%%%%%%%%%%%%%%%%%%%%%%%%%%%%%%%%%%%%%%%%%%%%%%%%%%%%%%%%%%%%%%%%%%%%%%%%

% 2-3 pages 
\section{Instantiation}
\label{sec:case}

This section presents practical experiments made by instantiating the deployment
methods with the JSFlow~\cite{JSFlow} information flow monitor. 

\subsection{Monitor}

JSFlow is a dynamic information flow monitor by tagging values with runtime
security labels used for security decisions. Whenever a potential security
violation has been encountered the monitor stops the execution with a security
error. 

JSFlow is implemented in JavaScript and supports full non-strict Ecma-262 (v5)
including the standard API and large parts of the browser specific execution
environment including the DOM.

The standard security policy of JSFlow tracks user information.  Any
information originating from the user is tagged and prevented from leaving the
browser. Other policies can be provided, e.g., by placing policy labels in the
HTML elements of the page. 


\subsection{Experiment scenarios}

Consider the scenario of a loan calculator service. To use the service the
user is required to input potentially sensitive information about income, 
loan amount, and loan term. The service then computes and displays, e.g., 
rate and monthly payment. Unless the calculation makes use of any trade secrets
it can be performed in its entirety on the client side; otherwise, the
user inputs must be sent back to the service provider that performs the
calculations and replies with the results. Even though it might be acceptable to 
the user to disclose the sensitive information the service provider,
doing so places a trust in the provider not to further disclose the information.

Below we investigate the different deployment methods on a calculator service
that sends information back to the service provider. In particular, we
investigate the loan and mortgage calculator \emph{mlcalc}. The user
information that is sent back to the server is, in turn, embedded in the return
page. There, the information is used to perform the loan calculation in
addition to being passed on to Google analytics, presumably for statistics. Now
assume that the user wants to prevent the information to be sent to Google.
Without server support there is no way for the JSFlow monitor to link the
values on the returned page to the values input by the user, regardless of
deployment method. On the other hand, if the server was information flow aware
it could tag the information originating from the user input on the returned
page. If this is done, the monitor can easily prevent the information from
being sent to Google analytics. Without server support, however, in order to
protect the user input we must prevent it from being sent back to mlcalc. 

\subsection{Security experiments}
%Jonas+Daniel

\todo{Daniel: what about the caches ets - where do we speak about those}
\paragraph{Browser extension}
%:
When the user loads mlcalc.com the extension traverses the page looking for
scripts to execute. After the scripts have executed the user of the page
enters the needed information and presses the calculate button. The extension
traps the button press event and uses the monitor to run the script associated
with the button. This script gathers the user information and tries to send
it back to mlcalc, which casues the montior to stop execution with a security 
error. At this point the user is presented with a security error message. 
Additionally, one could envision that the user is given the choice to
override the security error and allow the information to be sent.


\paragraph{Web proxy}

\todo{Jonas, fill in the web proxy results on mlcalc}


\paragraph{Service}

\todo{Jonas, fill in the service results on mlcalc}


\subsection{Performance experiments}
%Jonas

%1 page 
\section{Related work}
\label{sec:related}
We first discuss original work on reference monitors and their
inlining, then
inlining security checks in the context of
JavaScript, to relate to the architectures, 
and finally inlining for information-flow, to relate to our instantiations
of the architectures.

\paragraph{Inlined reference monitors}
Anderson~\cite{Anderson:72} introduces reference monitors and
outlines the principle requirements, recounted in Section~\ref{sec:intro}.
%
Erlingsson and Schneider~\cite{DBLP:conf/nspw/ErlingssonS99,Erlingsson:PhD04} instigate
the area of inlining reference monitors. 
This work studies both enforcement mechanisms and the policies
that they are capable of enforcing, with the focus on safety properties.
Inlined reference monitors
have been proposed in a variety of languages and settings: from
assembly code~\cite{DBLP:conf/nspw/ErlingssonS99} to Java~\cite{DBLP:conf/ecoop/DamJLP09,DBLP:journals/jcs/DamJLP10,DBLP:conf/ccs/DamGL12}.

Ligatti et al. \cite{Ligatti05editautomata:} present a 
general framework for security policies that can
be enforced by monitoring and modifying programs at runtime. 
They introduce \emph{edit automata} that enable
monitors to stop, suppress, and modify the behavior of programs. 



\paragraph{Inlining for secure information flow}
Language-based information-flow security~\cite{Sabelfeld:Myers:JSAC}
features work on inlining monitor for security information flow.
Secure information flow is not
a safety property~\cite{McLean:SSP94}, but can be approximated by
safety properties
(e.g.,~\cite{Boudol:FAST08,Sabelfeld:Russo:PSI09,Austin:Flanagan:PLAS09}).

Chudnov and
Naumann~\cite{Chudnov:Naumann:CSF10} have investigated an inlining
approach to monitoring information flow in a simple imperative language. They inline a flow-sensitive
hybrid monitor by Russo and
Sabelfeld~\cite{Russo:Sabelfeld:CSF10}. The soundness of the inlined
monitor is ensured by bisimulation of the inlined monitor and the
original monitor.

Magazinius et
al.~\cite{Magazinius+:SEC10,DBLP:journals/compsec/MagaziniusRS12} show
how to cope with dynamic code evaluation instructions by inlining
on-the-fly. 
Dynamic code evaluation instructions are
rewritten to make use of auxiliary functions that, when invoked at
runtime, inject security checks into the available string. 
The inlined code manipulates shadow variables to keep track of the
security labels of the program's variables.
See the discussion on the scalability of the shadow variable approach
in Section~\ref{sec:intro}.


\paragraph{Inlining for secure JavaScript}
Inlining has been explored JavaScript, although often focusing on
simple properties or preventing against fixed classes of vulnerabilities.
A prominent example in the context of the web is
BrowserShield~\cite{Reis+:TWeb07} by Reis et al. to instrument scripts with
checks for known vulnerabilities.

Yu et al.~\cite{Yu+:POPL07} and Kikuchi et
al.~\cite{DBLP:conf/aplas/2008} present an instrumentation approach
for JavaScript in the browser. Their framework allows instrumented
code to encode edit automata-based policies.

Phung et al.~\cite{DBLP:conf/ccs/PhungSC09} and 
Magazinius et al.~\cite{DBLP:conf/nordsec/MagaziniusPS10} develop
secure wrapping for self-protecting JavaScript. This approach is based
on wrapping built-in JavaScript methods with secure wrappers that
keep track of the security state and regulate access to the original built-ins.  



Agten et al.~\cite{DBLP:conf/acsac/AgtenABPDP12} present JSand, a
server-driven client-side sandboxing framework. The framework mediates
attempts of untrusted code to access to resources in the browser.  
In contrast to its predecessors such as
ConScript~\cite{DBLP:conf/sp/MeyerovichL10}, WebJail~\cite{DBLP:conf/acsac/AckerRDPJ11}, and Contego~\cite{DBLP:conf/trust/LuoD11},
the sandboxing is done purely at JavaScript level, requiring no
browser modification.

Despite the above progress on inlining security checks in JavaScript, achieving
Information-flow security for client-side JavaScript by inlining has been out of
reach for the current methods~\cite{Vogt+:NDSS07,DBLP:conf/pldi/ChughMJL09,Yip:Narula:Krohn:Morris:EUROSYS09,Jang+:CCS10,DeGroef+:CCS12}  that either
modify the browser or perform the analysis out-of-the-browser.

%1 page 
\section{Conclusions}
\label{sec:conc}
Different stakeholders have different interests in the security of web
applications. We have presented architectures for inlining security
monitors, to take into account the security goals of the users, system and
network administrators, and service providers and integrators.
%
We achieve great flexibility in the deployment options by considering
security monitors implemented as security-enhanced JavaScript interpreters.
%
The architectures allow deploying such a monitor in a browser
extension, web proxy, or web service.
%
We have reported on the security considerations and on the relative pros and
cons for each architecture.
%
We have applied the architectures to inline an information-flow
security monitor for JavaScript.
%
The security experiments show the
flexibility in supporting the different policies on the sensitive
information from the user. 
%
The performance experiments show reasonable overhead imposed by the
architectures and point out to the relative pros and cons from the
performance point of view. \todo{Make a bit more concrete when we have
the experimental results.}

Future work \todo{Daniel, Jonas: what's worth brining up here? Some
tracks that give us the credibility - pointing out a rich research agenda, rather
than pointing out the weaknesses of the current paper. :) }

% Less than 4 pages 
\bibliographystyle{plain}
\bibliography{literature}
\end{document}
